\documentclass[parskip=full]{scrartcl}
\usepackage[utf8]{inputenc} % use utf8 file encoding for TeX sources
\usepackage[T1]{fontenc}    % avoid garbled Unicode text in pdf
\usepackage[german]{babel}  % german hyphenation, quotes, etc
\usepackage{hyperref}       % detailed hyperlink/pdf configuration
\hypersetup{                % ‘texdoc hyperref‘ for options
pdftitle={SWT1: Lastenheftvorlage},%
bookmarks=true,%
}
\usepackage{graphicx}       % provides commands for including figures
\usepackage{csquotes}       % provides \enquote{} macro for "quotes"
\usepackage[nonumberlist]{glossaries}     % provides glossary commands
\usepackage{enumitem}

\makenoidxglossaries

\title{Specifications}
\author{}
\begin{document}

\maketitle

\section{Preface}

\section{Goal}
The goal is a software which performs image classification and is able to switch between deploy platforms and working modes.
It also should have a GUI to control the software and to show the results.

\section{Product use}
Image classification

\section{Acceptance criteria}
\subsection{Must}
\begin{itemize}[nosep]
\item [MAC010]: Image classification
\item [MAC020]: Running NN on heterogenous platforms, CPU and FPGA
\item [MAC030]: Different operating modes
\item [MAC040]: GUI for interacting with software
\item [MAC050]: Performance/power prediction
\end{itemize}

\subsection{Can}
\begin{itemize}[nosep]
\item [KAC060]: Training a nn for classification
\item [KAC070]: Illustration of a topology of the nn
\item [KAC080]: Object detection
\item [KAC090]: Choosing between different models
\item [KAC100]: Creating new models
\item [KAC110]: Voting of multiple nn
\item [KAC120]: Using video for classification
\item [KAC130]: Using camera for classification input
\item [KAC140]: Running NN on GPU
\end{itemize}

\section{Functional Requirements Must}
\begin{itemize}[nosep]
\item [MFR010]: Use neural network for image classification
\item [MFR011]: Deploy pre-trained nn with the corresponding layers
\item [MFR012]: Reading and parsing nn config/weight file
\item [MFR020]: Have high perfomance mode
\item [MFR021]: Have low power consumption mode
\item [MFR022]: Have high energy efficiency mode
\item [MFR023]: Calculator for power consumption
\item [MFR024]: Calculator for performance
\item [MFR025]: Dispatching the calculation process defined from the mode
\item [MFR030]: Support CPU for calculation
\item [MFR031]: Support FPGA for calculation
\item [MFR040]: Communication between Host-PC and platform
\item [MFR041]: Send image for classification
\item [MFR042]: Receive result
\item [MFR050]: GUI
\item [MFR060]: Showing results
\item [MFR070]: Choosing image for classification in format .jpg, .png, .bmp
\item [MFR080]: Choosing platform/hardware
\item [MFR081]: Look which devices are avaiable
\item [MFR082]: Testing communication
\item [MFR090]:	Choosing mode
\item [MFR091]: Show how much energy and power on which platform one needs/has
\end{itemize}

\section{Functional Requirements Can}
\begin{itemize}[nosep]
\item [KFR100]: Choosing between different models for example AlexNet, GoogleNet
\item [KFR110]: Train nn for classification of imageset (with transfer learning)
\item [KFR111]: Saving new trained nn (config an weights)
\item [KFR112]: Choosing/Reading data set
\item [KFR032]: Support GPU for calculation
\item [KFR113]: Backpropagation
\item [KFR114]: Choosing parameters like learning rate
\item [KFR120]: Illustrating nn topology
\item [KFR130]: Object detection algorithm
\item [KFR131]: Showing detected object
\item [KFR132]: Choosing between detection and classification mode
\item [KFR140]: Creating new topology
\item [KFR150]: Choosing between training and interference mode
\item [KFR160]: Choosing video in format .avi
\item [KFR161]: Apply classification for a certain amount of frames
\item [KFR170]: Connect with camera 
\item [KFR171]: Receive video stream from camera
\item [KFR180]: Detecting object
\item [KFR181]: Drawing bounding box 

\end{itemize}

\section{Productdata}
\begin{itemize}[nosep]
\item [PD010]: Images for classification
\item [PD020]: Labeled image set for training
\item [PD030]: Config/weight file of pretrained model
\end{itemize}

\section{Demarcation}
\begin{itemize}[nosep]
\item [D010]: No real time / no performance optimization
\item [D020]: No mobile support
\item [D030]: No neural network size optimization
\item [DO40]: No low-level (Assembler) optimization
\end{itemize}

\section{Non-functional requirements}
\begin{itemize}[nosep]
\item[NF10] 
\end{itemize}

\section{System models}

\subsection{Scenarios}
\subsection{Usecases}
\subsubsection{Seminarorganisation}

\end{document}
